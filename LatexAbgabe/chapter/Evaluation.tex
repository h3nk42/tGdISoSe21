\chapter{Fazit - Samuel }
Für ein ökologisch nachhaltiges Wirtschaften darf durch die Befriedigung der Bedürfnisse der heutigen Generation nicht die Bedürfnisbefriedigung zukünftiger Generationen beeinträchtigt werden. Die auf den vorhergegangenen Seiten auffgezeigten Fallbeispiele machen klar, dass die Blockchaintechnologie in seiner aktuellen und potenziellen Anwendung sowohl positiv, als auch negativ auf eine ökologisch nachhaltigere Welt einzahlen kann.

Die bekannteste Anwendung, Kryptowährungen, verursacht, vor Allem wenn sie auf dem POW Konsensmechanismus basiert, ökologische Probleme. Der Miningprozess des Bitcoin ist heute durch seinen Strombedarf potenziell für 56-58 MtCO2 pro Jahr verantwortlich. Außerdem bedarf die Kompetitivität im Mining der regelmäßigen Erneuerung der Mining-Hardware, sodass Schätzungen zufolge beim Austausch der Hardware zur nächsten Generation Elektromüll von 18 kt. entstehen wird

Auf der anderen Seite kann die Blockchain Lieferketten transparenter gestalten und somit Konsumenten die Möglichkeit geben, bei ihren Kaufentscheidungen auf eine nachhaltige Herkunft der Produkte zu achten. Außerdem könnten Blockchain-Netzwerke dafür genutzt werden, Strom aus erneuerbaren Energiequellen Peer-to-Peer zu handeln. Kleine Stromproduzenten könnten so am Energienetz teilnehmen und eine nachhaltige Energiewende vorantreiben.

Die Recherche und Analyse der Fallbeispiele zeigt auf, dass vor Allem POW Kryptowährungen ökologischen Schaden erzeugen, andere Blockchainanwendungen jedoch bei der Transformation zu ökologisch nachhaltigem Wirtschaften eine bedeutende Rolle spielen können.



