
\chapter{Analyse}
%ch




\section{Sender}
\subsection{Turm ausrichten}


\subsection{Analoges Signal senden}

\subsection{Schnittstelle Esp-32 und Arduino}

\begin{longtable}{|c|c|c|c|c|}
	\hline
	\multicolumn{1}{|c}{\textbf{Fall}} &
	\multicolumn{1}{|c}{\textbf{Pin 1}} &
	\multicolumn{1}{|c}{\textbf{Pin 2}} &
	\multicolumn{1}{|c}{\textbf{Pin 3}} &
	\multicolumn{1}{|c|}{\textbf{Pin 4}} \\
	\hline
	\endfirsthead
	
	\multicolumn{3}{c}{Beschreibung}\\ \hline
	\multicolumn{1}{|c}{\textbf{Spalte 1}} &
	\multicolumn{1}{|c}{\textbf{Spalte 1}} &
	\multicolumn{1}{|c}{\textbf{Spalte 1}} &
	\multicolumn{1}{|c}{\textbf{Pin 2}} &
	\multicolumn{1}{|c}{\textbf{Spalte 2}}\\
	\hline
	\endhead
	
	\multicolumn{2}{c}{Fortsetzung auf der nächsten Seite}
	\endfoot
	
	\caption{Pin-Falltabelle}
	\label{tab:example}
	\endlastfoot
	
	x zu weit rechts & 1 & 0 & / & / \\ \hline
	x zu weit links & 0 & 1 & / & /  \\ \hline
	x korrekt & 1 & 1 & / & /   \\ \hline
	y zu weit rechts & / & / & 1 & 0  \\ \hline
	y zu weit links & / & / & 0 & 1   \\ \hline
	y korrekt & / & / & 1 & 1  \\ \hline
\end{longtable}




\section{Empfänger}

\subsection{Analoges Signal verarbeiten}

\subsection{Empfangenes Signal ausgeben}

\subsection{Erkennungsmerkmal zur Hilfe der Zielsensorik}

